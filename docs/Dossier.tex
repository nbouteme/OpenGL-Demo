\documentclass[11pt, a4paper, titlepage]{article}

\usepackage[utf8]{inputenc}
\usepackage[T1]{fontenc}
\usepackage[french]{babel}
\usepackage{color}
\usepackage{url}
\usepackage{alltt}
\usepackage{listings}
\usepackage{fixltx2e}
\usepackage[hidelinks]{hyperref}

\frenchspacing
\setlength{\parskip}{0.7em}
\setcounter{secnumdepth}{0}
\begin{document}
\title{Rendu de scène sous OpenGL}
\author{Nabil Boutemeur, Cassian Assael, Alessandro Bonnafous}
\date{Octobre 2014}
\maketitle

\tableofcontents
\pagebreak

\section{Idée de base}
L'idée du projet était de fournir une démonstration de principes mathématiques appliqués
à un programme, et en même temps, montrer l'application d'objets mathématiques tels que les
matrices et les vecteurs en vrai, car bon nombre de ces conceptes sont applicables dans la réalité.

Nous voulons donc écrire qui montre les trois transformations de base possibles dans l'univers 3D:
\begin{itemize}
\item Translation
\item Rotation
\item Changement d'échelle
\end{itemize}
\subsection{Mise en oeuvre}

Après concertation, nous nous sommes décidés à faire une scène montrant trois objets avec des formes (plus ou moins) primitives.
Chacun de ces objets est ``activable'' à partir d'une touche de clavier, et provoquant pour chacun, l'animation d'une transformation.
\pagebreak

\subsection{Outils utilisés}

Nous utilisons l'outil \textbf{git} comme logiciel de gestion de version, il permet de mieux repérer les régressions introduites au fil
de la modification du code, et facilite le travail collaboratif, car chaque modification du code est associé a son auteur.

Le dépôt est disponible en libre accès sur \href{https://github.com/nbouteme/OpenGL-Demo}{{\color{blue}Github}}, sous license LGPLv3.
\textbf{Github} étend l'aspect collaboratif de git en fournissant un ensemble de service lié à un dépôt.

Cela comprend notamment le service d'intégration continue \href{https://travis-ci.org/nbouteme/OpenGL-Demo}{{\color{blue}Travis}}, qui, lorsqu'un 
changement se produit sur le dépôt va régénérer un environnement de développement et recompiler le projet, pour s'assurer que le 
code sur le dépôt principal sera toujours prêt à être déployé.

Il va aussi s'assurer que, lorsqu'un tiers veut contribuer du code au dépôt, la fusion du code existant et du code en attente d'intégration
soit toujours fonctionnel.

Concernant le programme lui-même, nous utilisons \textbf{cmake} comme système de build, le but du système de build étant de correctement lier entre
elle les différentes partie du projet et s'assurer que toutes les dépendances sont satisfaites. Le programme cmake génère alors un makefile qui
reconstruit les cibles dont au moins une dépendance aura été modifiée. La puissance de cmake réside aussi dans le fait qu'il n'est pas limité à générer
des makefile, mais aussi des fichiers de projet pour la plupart des IDE. Cela permet aux collaborateurs d'utiliser leur outil de choix pour modifier le code.

Ensuite, le programme utilise la version 3.3 du standard \textbf{OpenGL}.
Nous avons choisi cette version en particulier car c'est la version la plus répandue à l'heure actuelle. Elle a introduit des changements majeurs
dans l'API OpenGL ainsi que supprimée des fonctionnalités considérées comme néfaste pour un projet \footnote{\url{https://www.opengl.org/wiki/Legacy_OpenGL}}

Nous utilisons la bibliothèque \textbf{GLEW}, qui permet de charger à l'exécution du programme les fonctions OpenGL supportée par le pilote graphique, ainsi 
qu'éventuellement des extensions du standard, la bibliothèque \textbf{GLM} qui fournit une API permettant de manipuler certains types primitifs disponibles
dans le langage de programmation de shader \textbf{GLSL}, notamment les matrices et les vecteurs. La bibliothèque \textbf{SOIL} qui est une bibliothèque C très
légère, permettant le chargement et la décompression de format d'images. Et pour finir, la bibliothèque \textbf{GLFW}, elle aussi minimaliste, qui gère les 
entrées et les sorties du programme --- Fenêtre, souris, clavier, manette --- en plus de la création du contexte OpenGL.

\section{Conception du programme}

\subsection{Squelette du programme}

Cette section ne décrit que les classes du programme en surface.

\subsubsection{Singleton --- Classe Application}

Le but du singleton ici est de gérer un ensemble de ressources unique au programme.
Il est identique au singleton de Meyers, si ce n'est qu'un pointeur de fonction est utilisé pour récupérer son instance, au lieu de vérifier à chaque appel
si le singleton a déjà été instancié.

\lstset{language=C++}

\begin{lstlisting}
shared_ptr<Application> Application::createSingleton()
{
  Appplication::_getSingleton = &Application::returnSingleton;
  return shared_ptr<Application>(new Application());
}

shared_ptr<Application> Application::returnSingleton()
{
  return m_app->shared_from_this();
}
\end{lstlisting}

La fenêtre à afficher hérite d'une classe d'interface, pour modulariser le singleton et pouvoir instancier des fenêtres utilisant une autre API qu'OpenGL.

\subsubsection{Fenêtre --- Classe GLWindow}

Dans son constructeur, la fenêtre se contente d'appeler des fonctions d'initialisation OpenGL et GLFW. Puis la méthode run, s'occupe de gérer la boucle
principale, en créant une scène, et en l'affichant.

\subsubsection{Scène --- Classe Scene}

Une scène est un agrégat d'objet pouvant être dessinés, et pouvant être mis à jour (update) en plus d'une caméra.
Elle peut aussi contenir un ensemble d'effets de post-processing. --- Pas implémenté pour l'instant. 
On ne peut pas instancier une scene tel quel, la classe doit être dérivé, puis les objets qu'elle doit afficher sont instancié dans son constructeur.

\subsubsection{Modèles --- Classe Model}

Un modèle est un ensemble de triplé de sommet --- des faces ---, où chaque sommet a des attributs --- Normales et coordonnées de texture ---.
Dans le projet, un modelé est donc une figure géométrique \textbf{simple}

Il est intéressant de noter que j'emploie le terme \textbf{normale}, ici, avec le terme \textbf{sommet}, alors qu'une normale est un vecteur 
dénotant une direction perpendiculaire à une \textbf{face}. C'est parce que chaque sommet faisant parti d'un modèle peut faire parti de 
plusieurs faces, et on doit donc utiliser une normale différente pour un même dans le traitement d'une face distincte.

Lorsque des normales ne sont pas utilisée, la classe Model accepte un tableau d'éléments, en plus d'un tableau de sommets. Le tableau d'éléments permettra
alors de réutiliser des points déjà définis en utilisant leur indices, économisant ainsi de la VRAM et de la bande passante.

Dans la pratique, le tableau d'élément n'est jamais utilisé, car nous avons besoin des normales pour des applications plus intéressantes, notamment dans 
l'éclairage.

Pour en finir avec la classe Model, il faut aussi noter qu'elle peut prendre en paramètre le code source d'un fichier du format \textbf{OBJ}, qui est 
lu grâce à la classe OBJData.

\pagebreak
\subsubsection{Fichiers OBJ --- Classe OBJData}

La classe OBJData fait une analyse de la syntaxe du code qui lui a été fourni, et construit alors un tableau de sommet/attribut pret a l'emploi pour une
utilisation dans l'API OpenGL.

Du fait de la définition précédente de ``Modèle'' comme étant une figure géométrique simple, OBJData a été conçu seulement pour lire des fichiers OBJ basique.

Un fichier OBJ est défini par, dans l'ordre:
\begin{itemize}

\item une liste de sommets, tous séparé par un retour a la ligne:
  \begin{alltt}
    v \emph{x} \emph{y} \emph{z}
  \end{alltt}
  Où x, y et z sont des nombres à virgules flottantes.

\item une liste optionnelle de coordonnées de texture, tous séparé par un retour a la ligne:
  \begin{alltt}
    vt \emph{u} \emph{v}
  \end{alltt}
  Où u et v sont des nombres à virgules flottantes.

\item une liste optionnelle de normales, tous séparé par un retour a la ligne:
  \begin{alltt}
    vn \emph{x} \emph{y} \emph{z}
  \end{alltt}
  Où x, y et z sont des nombres à virgules flottantes.

\item une liste de faces, tous séparé par un retour a la ligne:
  \begin{alltt}
    f \emph{x\textsubscript{1}}/\emph{y\textsubscript{1}}/\emph{z\textsubscript{1}} \emph{x\textsubscript{2}}/\emph{y\textsubscript{2}}/\emph{z\textsubscript{2}} \emph{x\textsubscript{3}}/\emph{y\textsubscript{3}}/\emph{z\textsubscript{3}}
  \end{alltt}
  Où x\textsubscript{n}, y\textsubscript{n} et z\textsubscript{n} sont les indices des déclaration des Sommet/Coordonnées de texture/Normale et n
  le sommet constituant la face --- OpenGL, et par extension OBJData, veulent trois points par face.\\

\end{itemize}

Par exemple, un triangle rectangle, avec une seule normale perpendiculaire serait défini par le fichier suivant, prenant en compte trois définition
 possible d'une face:
\begin{alltt}
v  0.000000 1.000000 0.000000
v  1.000000 0.000000 0.000000
v  1.000000 1.000000 0.000000
vt 0.000000 1.000000
vt 1.000000 0.000000
vt 0.000000 0.000000
vn 0.000000 0.000000 1.000000

f 1/1/1 2/2/1 3/3/1 # si on utilise tout les attributs
f 1/1 2/2 3/3       # si on utilise sommet et textures
f 1//1 2//1 3//1    # si on utilise sommet et normales
\end{alltt}

\subsubsection{Caméra --- Classe Camera}

La caméra est une simple classe qui se contente de garder avec elle la matrice de vue et de projection, et gérer les entrées au clavier/souris/manette.

\subsubsection{Shader --- Classe Shader}

Un shader est un programme exécuté à divers niveaux de la pipeline graphique du GPU.
Ils ont l'avantage d'être très flexible, et surtout très puissant, car il bénéficient de la puissance de calcul parallèle monstrueuse du GPU.

Un shader n'est pas lié à un modèle en particulier, mais ils est simplement utilisé par OpenGL lorsqu'il effectue du dessin dans un tampon.

Un shader est le résultat de la compilation d'au moins 2 programmes de base, le vertex shader, qui effectue des calculs sommet par sommet,
notamment pour effectuer des transformation, selon les parametres qu'il lui auront été passé avant le dessin, et le pixel shader, qui va
 donner une couleur à un pixel en fonction des paramètres qu'il lui auront été donné avant le dessin, ou récupéré directement du vertex shader, qui
s'exécute avant le pixel shader dans la pipeline.

Il existe aussi des geometry shaders, qui permettent notamment de générer de manière procédurale des objets tridimensionnels, et
 --- à partir d'OpenGL 4.0 --- les compute shaders qui sont utilisé pour des calculs plus généraux, comme par exemple des 
simulations d'évènements physiques comme des tissus, et le couple Tesselation Control/Evaluation shader qui prend une géométrie existante, en génère des sommets
puis applique a ces nouveau sommets des transformations supplémentaire, ils sont surtout utilisé pour augmenter le niveau de détail d'un objet.

La classe Shader s'occupe donc de la vie d'un shader, de sa création à sa destruction.
Elle combine un vertex shader et un fragment shader, et accessoirement un geometry shader, et les combine en un seul shader prêt a l'emploi. 

\subsubsection{Gestion des ressources}

Un des problèmes qui s'est posé pendant le développement était : comment est-ce que tout ces modèles, shader, texture, etc, allais être chargé dans le programme ?
Est-ce que les garder dans un fichier était une solution idéale ? Le programme serait contraint de se trouver dans un répertoire à partir d'où il avait accès 
à toutes ses ressources, avec les bonnes permissions, et un déplacement ou une copie de l'exécutable s'accompagne de tout ses fichiers.
Une première solution avait était d'utiliser le linker ld, qui est capable de transformer n'importe quel fichier en .o, pouvant être linker statiquement dans
 le programme, cela posait trois autre problèmes, le nom de symboles généré par ld n'était pas paramétrable, et donnait trois symboles sous la forme
 ``\_binary\_nom\_extension\_'' suivis de start, end, et size. Ensuite le problème était que pour chaque fichier ajouté, il fallait ajouter trois symboles au code du 
programme, et c'est vite devenu pénible à gérer, et dernièrement, le symbole cense representer la taille, ne contanait pas vraiment la taille, mais son adresse 
elle même était la taille d'une ressource.

Au final, un format de fichier RES d'archivage sans compression a été conçu pour répondre à ce besoin, ainsi qu'une classe capable de manipuler ce type
 de fichier, situé dans une bibliothèque séparée et un exécutable genere les archives, et ld n'a plus qu'a linker ces archives avec l'exécutable final, 
nécessitant alors seulement de modifier le code du programme non plus a chaque ajout de fichier, mais a chaque ajout de répertoire, ce qui n'arrive pas souvent.

\paragraph{Anatomie d'un fichier RES}

Le header d'un fichier RES défini de la manière suivante.

Le header commence par les 3 caracteres ASCII ``RES'' suivi d'un octet présentant la version du format d'archive --- typiquement 1 --- suivi du nombre de
ressource contenue dans le ficher, encodé en big endian, sur 32 bits.
Une archive valide vide a donc une taille de 8 octets.

Ensuite le header contient une table de ressource, de taille variable, car chacune de ses entrées a une taille variable.
Une entrée dans la table est typiquement la suivante, en format hexadécimal:

\begin{alltt}
[ aa aa aa aa ] [ bb {... a fois ...} bb]
[ cc cc cc cc ] [ dd dd dd dd ]
\end{alltt}

Où aa aa aa aa est un entier 32 bit non signé en big endian représentant la taille du nom de la ressource, bb ... bb est le nom de la ressource,
cc cc cc cc est un entier 32 bit non signé en big endian représentant la taille de la ressource elle même, et enfin dd dd dd dd est
un entier 32 bit non signé en big endian représentant l'adresse du premier octet de la ressource dans  l'archive.

Enfin, le header est constituer d'octets à 0 jusqu'à être aligné avec le segment qui suit --- padding.

\paragraph{Gestionnaire de Ressources --- Classe ResourceManager}

La classe ResourceManager s'en tient au spécifications données précédemment pour permettre de lire, gérer et sauvegarder une archive. 

\paragraph{Générateur de Ressources  --- Programme resgen}

Le programme resgen prend une liste de fichiers en paramètre, et en créer une archive, écrite sur la sortie standard.
Si un des fichier est une archive RES, alors elle n'est pas ajouter a l'archive en cours de création mais ses ressources le sont.
Comme l'archive résultante est ecrite sur la sortie standard, vous sauvegarder une archive <<The Unix Way>>, avec la redirection de votre shell:
 
\begin{alltt}
\$> ./resgen file1 file2 >  ar.res
\end{alltt}

\subsection{Utilisation de la ``nouvelle'' API d'OpenGL}

\subsubsection{Modèles}

\subsubsection{Shaders}

\subsubsection{Effets implémentés}

\end{document}
